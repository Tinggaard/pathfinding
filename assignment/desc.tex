\documentclass[a4paper, 12pt]{article}
\usepackage[danish]{babel} % Language
\usepackage[utf8]{inputenc} % Encoding
\usepackage[pdftitle={Projektbeskrivelse},pdfauthor={Jens Tinggaard}, hidelinks, bookmarks=true]{hyperref} %References
\usepackage[margin=1in]{geometry} %Margins


\title{%
Projektbeskrivelse \\
\large Programmering B
}
\author{Jens Tinggaard, 3.E}
\date{\today}


\setlength{\parindent}{0em} % Identeringsstørrelse = 0


\begin{document}
    \maketitle

    Jeg vil gerne lave et program, der er i stand til at illustrere en pathfinding algoritme. Jeg vil gerne vise fordelen ved en pathfinding algoritme, dette kunne f.eks. gøres ved at sammenligne med en bruteforce algoritme. Umiddelbart vil det tage udgangspunkt i labyrinter, hvor jeg vil vise hvordan de forskellige algoritmer virker.
    \paragraph{Algoritmer til overvejelse}
    \begin{itemize}
        \item Dijkstra (evt. A*)
        \item Breadth-first
        \item Depth-first
        \item Vægfølgende
    \end{itemize}
    Dette er ikke en endelig liste, den afhænger meget af hvor svær opgaven viser sig at være. Hvordan labyriten bliver til et input er heller ikke fastsat... Om det er et billede eller en tekstfil med tegn i eller måske begge dele, afhænger af hvad jeg finder nemmest.

    \paragraph{Funktionalitet af programmet}
    \begin{itemize}
        \item Implementering af en pathfinding algoritme
        \item Kunne løse en labyrint
        \item Sammenligning med andre metoder til at løse en labyrint
        \item Visuel repræsentation af løsningen
    \end{itemize}

    \paragraph{Evt. udviddelser}
    \begin{itemize}
        \item Visualisering af hastighed vs. størrelse af labyrint for forskellige metoder
        \item Implementering af flere algoritmer (nogle af dem fra ovenstående liste)
    \end{itemize}

\end{document}
